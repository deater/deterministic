\section{Conclusions and Future Work}
In our experiments we have found only a small minority of
x86\_64 events to be deterministic and without overcount:
{\tt RETIRED\_STORES} on Core2 and 
{\tt BR\_INST\_RETIRED\_CONDITIONAL} on SandyBridge and Westmere.
This lack of useful events limits 
the use of performance counters for advanced applications such
as deterministic replay and threading libraries on the popular x86\_64
architecture.  

Many potentially deterministic events are rendered unusable
by including the unpredictable hardware interrupt count.
This can be mitigated by subtracting off a separate
interrupt counter event (if available),
but this will not help in the deterministic
use case where exact overflow is desired in order to 
stop at precise locations.  

Our investigation of other architectures shows that
deterministic events are more common on non-x86 hardware.
This shows that deterministic events can be accomplished
and are not an unsolvable problem.  Unfortunately 
these platforms are typically not available to most
users.

New users of performance counters are often frustrated that the results
they measure are not the ones they ``know'' to be correct.
Eventually the users learn the sources of the error, and undertake 
analysis that allows for run-to-run variation in the results.
It becomes almost a rite of passage, learning why the counters work
the way they do, and working around them.  This fatalistic view
of the quality of counters explains the lack of
impetus for fixing the underlying problem.

We propose that there are definite benefits to providing deterministic
counters with little overcount or variation.  
Existing methodologies that can stand some variation will not be harmed, 
and new and better uses for the counters will be found.  
Use of counters by non-experts can then be encouraged, 
as there will be so many fewer caveats to their use.

The various x86\_64 vendors need to be strongly encouraged
to fix the performance monitoring units on their respective CPUs.
There are many inherent hardware problems with providing deterministic
counters, but other non-x86 architectures seem to have solved them.
This may mean simplifying the available counters or limiting the
number of available events, but in practice few people use the counters
at all, let alone the full feature set.

A change like this will not happen overnight;
In the meantime more work on analyzing the causes and amounts
of variations can be done.  Manually generating and validating
test suites is a slow, tedious process.  We are investigating
a method of automated testcase generation and validation that
can vastly improve the process.

When deterministic counters do become available, they will
be welcomed not only by those working on deterministic replay and simulator
validators, but also by all users of performance counters.  


