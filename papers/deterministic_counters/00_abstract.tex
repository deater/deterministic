Ideal hardware performance counters provide exact deterministic results.
Real-world performance monitoring unit (PMU) implementations
do not always live up to this ideal.
Events that should be exact and deterministic 
(such as retired instructions) show run-to-run 
variation and overcount on x86\_64 machines, even when 
run in strictly controlled environments. 
These effects are non-intuitive to casual users and 
cause difficulties when strict determinism is desirable,
such as when implementing
deterministic replay or deterministic threading libraries.

We investigate eleven different x86\_64 CPU implementations and
discover the sources of divergence from expected count totals.  
Of all the counter events investigated, we find only a few
that exhibit enough determinism to be used without adjustment in
deterministic execution environments.  
We also briefly investigate ARM, IA64, POWER and SPARC systems
and find that on these platforms the counter events have more
determinism.  

We explore various methods of working around the limitations 
of the x86\_64 events, but in many cases this is not possible
and would require architectural redesign of the underlying
PMU.
